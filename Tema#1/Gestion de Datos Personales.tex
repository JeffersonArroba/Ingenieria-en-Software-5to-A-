Introduccion
En la actualidad, la gestión de datos personales es un tema cada vez más relevante en la era digital en la que vivimos. El avance de la tecnología y su integración en prácticamente todos los aspectos de nuestra vida ha generado una gran cantidad de información personal que es recopilada, compartida y almacenada en diversas plataformas y contextos.
Sin embargo, esta situación también ha planteado preocupaciones sobre la privacidad, la seguridad y el uso indebido de los datos personales. Por ello, es necesario establecer un enfoque sólido para la gestión responsable de los mismos. En este sentido, individuos, organizaciones y autoridades deben encontrar un equilibrio para aprovechar los beneficios de los datos en la toma de decisiones, a la vez que se protegen los derechos fundamentales de privacidad y autonomía de las personas.
En este proyecto se aborda la gestión de datos personales en un mundo interconectado y digitalizado. Se analizan las prácticas actuales en la recopilación, almacenamiento y procesamiento de datos personales, así como los marcos legales y las regulaciones que buscan proteger la privacidad de los individuos. Además, se consideran las implicaciones éticas y sociales de la gestión de datos y se examinan tanto los riesgos como las oportunidades que se presentan para las personas y las organizaciones.

A lo largo del proyecto se realizan análisis comparativos de enfoques de gestión de datos en diferentes sectores y geografías, se examinan estudios de caso y se proponen recomendaciones para una gestión más efectiva y responsable de los datos personales. De esta forma, se busca comprender las complejidades del tema y estar preparados para abordar los desafíos que surgen en la era de la información, garantizando el respeto de los derechos individuales en una sociedad cada vez más digital.

Objetivos Generales
El propósito central de este proyecto es analizar, evaluar y proponer estrategias efectivas para la gestión responsable de datos personales en un entorno digital en constante evolución. Para ello, se adopta un enfoque integral que busca comprender las implicaciones legales, éticas, técnicas y sociales asociadas con la recopilación, almacenamiento, procesamiento y uso de datos personales, que garantice la privacidad, la seguridad y el cumplimiento normativo, promoviendo la conciencia sobre los derechos de los individuos y fomentando prácticas éticas y responsables en la recopilación, almacenamiento y uso de datos personales en entornos digitales y organizacionales.



Objetivos Especificos
Investigar el marco legal y normativo: Se analizan las leyes y regulaciones vigentes relacionadas con la gestión de datos personales en el ámbito nacional e internacional para identificar las obligaciones legales que deben cumplir las organizaciones en cuanto a la recopilación, almacenamiento, procesamiento y protección de estos datos.

Evaluar políticas de privacidad: Se examinan las políticas de privacidad de una variedad de organizaciones y plataformas digitales para identificar la claridad y la transparencia de estas políticas, así como la accesibilidad y comprensión por parte de los usuarios.

Realizar un análisis de riesgos: Se identifican los posibles riesgos y vulnerabilidades asociados con la gestión de datos personales, incluyendo la posibilidad de brechas de seguridad, filtración de información sensible y el riesgo de discriminación algorítmica.

Desarrollar estrategias de seguridad de datos: Se diseñan y proponen estrategias y medidas de seguridad de datos, incluyendo la encriptación, autenticación de usuarios, sistemas de detección de intrusiones y procedimientos de respuesta ante incidentes de seguridad.

Crear materiales educativos: Se elaboran materiales educativos y de concienciación dirigidos a los usuarios que expliquen de manera clara y accesible los conceptos clave de la gestión de datos personales, sus derechos y cómo pueden proteger su privacidad en línea.



